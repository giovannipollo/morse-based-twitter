 \documentclass[a4paper]{article}
\usepackage{graphicx}
\usepackage{titling}
\usepackage{amsmath}
\usepackage{hyperref}

\title{\Huge Assignment OSEK \\ \LARGE Trampoline \& Arduino}
\author{\huge Giovanni Pollo \\ \\ \huge 290136}
\date{}
\renewcommand\maketitlehooka{
  \begin{center}
    \includegraphics[width=0.65\textwidth]{Immagini/LogoPolito.png}
  \end{center}%
}

\begin{document}
\begin{titlepage}
  \centering
  \vspace{2px}
\end{titlepage}
\maketitle


\pagenumbering{gobble} % remove page number for the title


\newpage

\pagenumbering{arabic} % Start page number from 1

\section{Structure \& Algorithm}
The structure chosen is based on a single extended task. The main reason is simplicity. Thanks to a single task, the solution is as straightforward as possible, and the memory occupation is also very low.
\newline

To guarantee the system's timing, I opted for an event triggered every \(100ms\). \\
The conversion is softcoded, thanks to the global variable \emph{LED}. The external loop reads all sentences, while the internal one analyzes every single letter. The program compares every character against 'A', and it computes the value of \emph{pos}. \\
Thanks to the variable \emph{pos}, we can get the morse code of the analyzed letter, convert it into a sequence of 0 and 1, and then save the output inside the variable \emph{LED} thanks to the \textbf{populateLED()} function. \\
Another critical method is the \textbf{string\_lenght()} one, which has been included to keep the code as general as possible. Furthermor, by not hardcoding the sentences' length, we can save a little bit of data memory.
\newline

To implement the \(180s\) pause, I opted for a counter (variable \emph{cnt}) which maximum value is 1800. In fact:
\begin{equation}
  max\_cnt\_value = \frac{pause\_time}{event\_time} = \frac{180\ s}{0.1\ s} = 1800
\end{equation}
Similarly, the \(0.5s\) pause uses a counter in which the maximum value is \(5\).
\newline

The last thing to mention is \textbf{STACKSIZE}. The minimum value I found in order to have a correct output is 112. Because 112 is not a power of 2, I decide to use 128.
\section{Timing \& Errors}
\subsection{Timing} \label{Timing}
As explained in the first paragraph, the code has a periodic alarm (every \(100ms\)) that activates an event. The main problem is that the Trampoline \emph{SystemCounter} is the same as the \emph{Systick} used in Arduino, which counts a tick every \(1024 \mu s\). To obtain \(100ms\) period, the value assigned to  \textbf{CYCLETIME} must be:
\begin{align*}
  CYCLETIME & = \frac{event\_time}{tick\_time} = \frac{100ms}{1024\mu s} = \frac{100 \cdot 10^{-3} s}{1024 \cdot 10^{-6} s} \\[0.5ex]
            & = 97.65625 \approx 98
\end{align*}
The choice for CYCLETIME is, therefore, 98. We can now compute what the real value for \(100ms\) is:
\begin{equation*}
  real\_100ms = 98 \cdot 1024\mu s = 100.352ms
\end{equation*}
From this, it is easy to evaluate the default error:
\begin{align}
  Default\_Error & = \frac{real\_100ms - 100ms}{100ms}                  \label{Default_Error1} \\[0.5ex]
  Default\_Error & = \frac{100.352ms - 100ms}{100ms} = 0.352\%    \label{Default_Error2}
\end{align}

\subsection{Errors}
To analyze the program's timing errors, I used the Arduino function \textbf{micros()}. We can identify three errors:
\begin{enumerate}
  \item \(100ms\): I obtained \(0.352\%\) error
        \begin{align*}
          Error & = \frac{value\_with\_micros - ideal\_value}{ideal\_value} \\[0.5ex]
          Error & = \frac{100352\mu s - 100000\mu s}{100000\mu s} = 0.352\%
        \end{align*}
  \item \(500ms\): I obtained \(0.352\%\) error
        \begin{align*}
          Error & = \frac{value\_with\_micros - ideal\_value}{ideal\_value} \\[0.5ex]
          Error & = \frac{501760\mu s - 500000\mu s}{500000\mu s} = 0.352\%
        \end{align*}
  \item \(180s\): I obtained \(0.352\%\) error
        \begin{align*}
          Error & = \frac{value\_with\_micros - ideal\_value}{ideal\_value}          \\[0.5ex]
          Error & = \frac{180633600\mu s - 180000000\mu s}{180000000\mu s} = 0.352\%
        \end{align*}
\end{enumerate}

It's easy to notice that all the errors are the same, and they are all coeherent with the \emph{Default\_Error} evalueted in section \ref{Timing}, with the equations \ref{Default_Error1} and \ref{Default_Error2}


\section{Memory Occupation}

To analyze memory occupation, I compared my solution with a blank code (an empty PeriodicTask triggered every \(100ms\)).

\begin{table}[h]
  \centering
  \begin{tabular}{|| c || c || c || c ||}
    \hline
    \textbf{Text} & \textbf{Data} & \textbf{Bss} & \textbf{Dec} \\
    \hline
    5730 Bytes    & 278 Bytes     & 382 Bytes    & 6390 Bytes   \\
    \hline
  \end{tabular}
  \caption{Blank code memory occupation}
  \label{Table1}
\end{table}


\begin{table}[h]
  \centering
  \begin{tabular}{|| c || c || c || c ||}
    \hline
    \textbf{Text} & \textbf{Data} & \textbf{Bss} & \textbf{Dec} \\
    \hline
    7034 Bytes    & 678 Bytes     & 260 Bytes    & 7972 Bytes   \\
    \hline
  \end{tabular}
  \caption{My solution memory occupation}
  \label{Table2}
\end{table}

By comparing \autoref{Table1} and table \autoref{Table2}, it's easy to see the benefit of the online translation. Because the program translates letter by letter, the data occupation doesn't increase too much. As a downside, the text size grows slightly, but because Arduino RAM is limited to  \(2kB\), I decided to optimize the program to use the least amount of RAM.


\end{document}